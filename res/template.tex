\documentclass[a4paper,11pt, notitlepage]{article}
\usepackage[T1]{fontenc}
% \usepackage{polski}
\usepackage[top=15mm, bottom=15mm, left=20mm, right=20mm]{geometry}
\usepackage[utf8]{inputenc}
\usepackage{listings}
\usepackage{geometry}
\usepackage{graphicx}
\usepackage{amsfonts}
\usepackage{tikz}
\usepackage{hyperref}
\usepackage{amsmath, amsthm, amssymb}
\usepackage[linesnumbered, algoruled]{algorithm2e}
\usepackage{pdfpages}
\usepackage{appendix}
%opening

%\setlength{\textwidth}{16 cm}
%\setlength{\textheight}{25 cm}
%\setlength{\evensidemargin}{-1.5 cm}
%\setlength{\oddsidemargin}{-0 cm}
%\setlength{\topmargin}{-1 cm}

\newcommand{\cA}{{\mathcal A}}
\newcommand{\cO}{{\mathcal O}}
\newcommand{\cW}{{\mathcal W}}
\newcommand{\cM}{{\mathcal M}}
\newcommand{\cE}{{\mathcal E}}
\newcommand{\cF}{{\mathcal F}}
\newcommand{\cI}{{\mathcal I}}
\newcommand{\cD}{{\mathcal D}}
\newcommand{\cR}{{\mathcal R}}
\newcommand{\cS}{{\mathcal S}}
\newcommand{\cB}{{\mathcal{B}}}
\newcommand{\cC}{{\mathcal{C}}}

\newcommand{\wadapt}{weakly-adaptive}
\newcommand{\noadapt}{non-adaptive}
\newcommand{\chainadapt}{$k$-chain-ordered}
\newcommand{\kthick}{$k$-ordered}
\newcommand{\sadapt}{strongly-adaptive}
\newcommand{\ordered}{ordered}
\newcommand{\oblivious}{oblivious}
\newcommand{\constrained}{constrained}
\newcommand{\loa}{linearly-ordered}
\newcommand{\safba}{strongly-adaptive $f$-bounded}
\newcommand{\fba}{strongly-adaptive $f$-bounded}
\newcommand{\walba}{weakly-adaptive linearly-bounded}
\newcommand{\salba}{strongly-adaptive linearly-bounded}
\newcommand{\wafba}{weakly-adaptive $f$-bounded}
\newcommand{\lba}{linearly-bounded}
\newcommand{\ua}{unbounded}
\newcommand{\rd}{RD}
\newcommand{\rounddelay}{round-delay}
\newcommand{\adapt}{adaptive $f$-bounded}
\newcommand{\ofba}{ordered $f$-bounded}
\newcommand{\lofba}{linearly-ordered $f$-bounded}
\newcommand{\kcofba}{$k$-chain-ordered $f$-bounded}
\newcommand{\ktofba}{$k$-thick-ordered $f$-bounded}
\newcommand{\acofba}{anti-chain-ordered $f$-bounded}
\newcommand{\crdfba}{$c$-round-delay $f$-bounded}
\newcommand{\nafba}{non-adaptive $f$-bounded}

\newcommand{\DA}{Do-All}
\newcommand{\CO}{consensus}
\newcommand{\polylog}{{\mbox{polylog}}}


\title{Umowa najmu lokalu mieszkalnego}
\author{}
\date{}

\begin{document}



\maketitle

Zawarta w dniu [15] we Wrocławiu pomiędzy:
[1], zamieszkały w [2], [3], legitymujący się dowodem osobistym nr [4], PESEL [5], zwanym dalej Wynajmującym, a
[6], zamieszkały/a w [7], [8] [19] legitymujący/a się dowodem osobistym nr [9], PESEL [10], zwanym dalej Najemcą o następującej treści:
§ 1.
    1. Wynajmujący oświadcza, że jest właścicielem lokalu mieszkalnego nr [11] o powierzchni użytkowej [12] m2, położonego 	w: [13], [14].
§ 2.
    1. Wynajmujący z dniem [15] oddaje Najemcy w najem pokój w lokalu określonym w § 1 pkt 1 wraz z wyposażeniem z przeznaczeniem na cele mieszkalne.
 	
    2. Wynajmujący przekaże Najemcy klucze do lokalu - jeden komplet w dniu oddania lokalu w najem. 
§ 3.
    1. Umowa zostaje zawarta na okres od [15] do [16].
 	
    2. Wynajmujący ma prawo jednostronnego rozwiązania umowy w trybie natychmiastowym, jeżeli Najemca:
 	
        a. pomimo upomnienia nadal używa lokalu w sposób sprzeczny z umową lub niezgodnie z jego przeznaczeniem, lub zaniedbuje obowiązki, dopuszczając do powstania szkód, lub niszczy urządzenia	przeznaczone do wspólnego korzystania przez mieszkańców albo wykracza w sposób rażący lub uporczywy przeciwko porządkowi domowemu, czyniąc uciążliwym korzystanie z innych lokali, 
 		
        b. jest w zwłoce z zapłatą czynszu lub innych opłat za używanie lokalu za jeden pełny okres płatności (to jest jeden miesiąc) 	pomimo uprzedzenia go o zamiarze wypowiedzenia stosunku prawnego i wyznaczenia dodatkowego, terminu do zapłaty zaległych i bieżących należności, 		
 		
        c. wynajął, podnajął albo oddał do bezpłatnego używania lokal lub jego część bez wymaganej zgody właściciela.
 	
    3. Najemca ma prawo jednostronnego rozwiązania umowy w przypadku niemożności używania lokalu, lub istotnego utrudnienia w używaniu lokalu zgodnie z umową, po uprzednim pisemnym zawiadomieniu Wynajmującego.

§ 4.
    1. Najemca zobowiązany będzie do zapłaty Wynajmującemu czynszu najmu w wysokości:
[17] złotych miesięcznie od osoby (słownie: [18] złotych).
    2. Zapłata następować będzie z góry w terminie do 7 dnia każdego 	miesiąca obowiązywania umowy na konto Wynajmującego.
 	
    3. Pierwsza wpłata zostanie dokonana w dniu podpisania umowy.
 	
    4. Strony ustalają kaucję w wysokości jednomiesięcznego czynszu.
 	
    5. W związku z § 4 pkt 4 Najemca zobowiązuje się pokryć ewentualne roszczenia z tytułu zniszczenia lokalu ponad normalne zużycie w trakcie używania. Wysokość roszczeń zostanie ustalona na podstawie faktur lub rachunków powykonawczych usuwających zniszczenia.
 	
    6. Najemca pokrywa koszty eksploatacji lokalu obejmujące:
 	
        a. opłaty za gaz - płatne według wskazań liczników,
        b. opłaty za centralne ogrzewanie - płatne według wskazań liczników,
        c. opłaty za energię elektryczną - płatne według wskazań liczników,
        d. opłaty za wodę - płatne według wskazań liczników,
 	
    7. Należności wymienione w § 4 pkt 6 Najemca będzie płacić 	Wynajmującemu na podstawie faktur lub rachunków wystawionych przez dostawców, w terminie do siedmiu dni roboczych od daty doręczenia przez Wynajmującego Najemcy faktur lub rachunków.
 	
    8. W przypadku zwłoki w płatności czynszu i opłat eksploatacyjnych Najemca jest obowiązany do zapłaty odsetek za zwłokę w ustawowej wysokości oraz odsetek stosowanych przez dostawców mediów.
§ 5.
    1. Najemcy nie wolno dokonywać zmiany przeznaczenia lokalu.
 	
    2. Najemcy nie wolno oddawać lokalu w podnajem lub do bezpłatnego używania.
 	
    3. Ewentualne planowane przez najemcę adaptacje lub przebudowa pomieszczeń muszą 	być za każdym razem uzgadniane z Wynajmującym i wymagają jego 	pisemnej zgody.
 	
    4. Najemca może wprowadzić w lokalu ulepszenia tylko za pisemną zgodą Wynajmującego.
§ 6.
    1. Remonty o charakterze generalnym obciążają Wynajmującego.
 	
    2. Najemca jest obowiązany do dokonywania na własny koszt konserwacji i napraw niezbędnych do zachowania przedmiotu najmu w stanie niepogorszonym.
 	
    3. Najemca zobowiązany jest poinformować Wynajmującego niezwłocznie o konieczności przeprowadzenia napraw i innych czynności, które zgodnie z obowiązującymi przepisami obciążają Wynajmującego.
 	
    4. Najemca zobowiązany jest poinformować Wynajmującego niezwłocznie o konieczności wymiany wkładki do drzwi wejściowych w przypadku 	zgubienia lub zniszczenia któregokolwiek kompletu kluczy, który Najemca otrzymał w dniu podpisania umowy od Wynajmującego.
 	
    5. Najemca zobowiązany jest udostępnić Wynajmującemu lokal do wglądu.
§ 7.
    1. Po ustaniu stosunku najmu Najemca zobowiązany jest niezwłocznie zwrócić przedmiot najmu Wynajmującemu w stanie niepogorszonym.
 	
    2. Najemca nie ponosi odpowiedzialności za zmiany i zniszczenia w 	lokalu powstałe na skutek naturalnego zużycia.
§ 8.
Wynajmujący nie odpowiada za szkody wyrządzone ruchomościom Najemcy w wyniku zalania, ognia, włamania i innych zdarzeń losowych.
§ 9.
W sprawach nieuregulowanych niniejszą umową zastosowanie mają przepisy Kodeksu cywilnego oraz ustawy o ochronie praw lokatorów.
§ 10.
Zmiana niniejszej umowy wymaga formy pisemnej pod rygorem nieważności.
§ 11.
Umowę sporządzono w pięciu jednakowo brzmiących egzemplarzach, cztery egzemplarzy dla Najemcy i jeden dla Wynajmującego.

NAJEMCA


[6] [20]


WYNAJMUJĄCY


[1] [21]




\end{document}



