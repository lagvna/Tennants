\documentclass[a4paper,11pt, notitlepage]{article}
\usepackage[T1]{fontenc}
% \usepackage{polski}
\usepackage[top=15mm, bottom=15mm, left=20mm, right=20mm]{geometry}
\usepackage[utf8]{inputenc}
\usepackage{listings}
\usepackage{geometry}
\usepackage{graphicx}
\usepackage{amsfonts}
\usepackage{tikz}
\usepackage{hyperref}
\usepackage{amsmath, amsthm, amssymb}
\usepackage[linesnumbered, algoruled]{algorithm2e}
\usepackage{pdfpages}
\usepackage{appendix}
\usepackage{sectsty}
\sectionfont{\centering}
\title{Umowa najmu lokalu mieszkalnego}
\author{}
\date{}
\begin{document}
\maketitle
Zawarta w dniu 1.10.2020 we Wrocławiu pomiędzy:
Jarosław Mirek, zamieszkały w Zielona 3, 58-200 Dzierżoniów legitymujący się dowodem osobistym nr AA112233, PESEL 90020512132, zwanym dalej \textit{Wynajmującym}, a
Jarosław Mirek, zamieszkały/a w Południowa 23, 50-384 Wrocław legitymujący/a się dowodem osobistym nr AAA001122, PESEL 90020512445, zwanym dalej \textit{Najemcą} o następującej treści:
\section*{§ 1.}
    \begin{enumerate}
        \item Wynajmujący oświadcza, że jest właścicielem lokalu mieszkalnego Reja 3 o powierzchni użytkowej 54 m2, położonego w: Wrocław.
    \end{enumerate}
\section*{§ 2.}
    \begin{enumerate}
     \item Wynajmujący z dniem 1.10.2020 oddaje Najemcy w najem pokój w lokalu określonym w § 1 pkt 1 wraz z wyposażeniem z przeznaczeniem na cele mieszkalne.
    \item Wynajmujący przekaże Najemcy klucze do lokalu - jeden komplet w dniu oddania lokalu w najem.
    \end{enumerate}
\section*{§ 3.}
    \begin{enumerate}
     \item Umowa zostaje zawarta na okres od 1.10.2020 do 30.07.2020.
     \item Wynajmujący ma prawo jednostronnego rozwiązania umowy w trybie natychmiastowym, jeżeli Najemca:
        \subitem a) pomimo upomnienia nadal używa lokalu w sposób sprzeczny z umową lub niezgodnie z jego przeznaczeniem, lub zaniedbuje obowiązki, dopuszczając do powstania szkód, lub niszczy urządzenia	przeznaczone do wspólnego korzystania przez mieszkańców albo wykracza w sposób rażący lub uporczywy przeciwko porządkowi domowemu, czyniąc uciążliwym korzystanie z innych lokali,
        \subitem b) jest w zwłoce z zapłatą czynszu lub innych opłat za używanie lokalu za jeden pełny okres płatności (to jest jeden miesiąc) 	pomimo uprzedzenia go o zamiarze wypowiedzenia stosunku prawnego i wyznaczenia dodatkowego, terminu do zapłaty zaległych i bieżących należności,
        \subitem c) wynajął, podnajął albo oddał do bezpłatnego używania lokal lub jego część bez wymaganej zgody właściciela.
        \item Najemca ma prawo jednostronnego rozwiązania umowy w przypadku niemożności używania lokalu, lub istotnego utrudnienia w używaniu lokalu zgodnie z umową, po uprzednim pisemnym zawiadomieniu Wynajmującego.
    \end{enumerate}
\section*{§ 4.}
    \begin{enumerate}
        \item Najemca zobowiązany będzie do zapłaty Wynajmującemu czynszu najmu w wysokości:
600 złotych miesięcznie.
        \item Zapłata następować będzie z góry w terminie do 7 dnia każdego miesiąca obowiązywania umowy na konto Wynajmującego.
        \item Pierwsza wpłata zostanie dokonana w dniu podpisania umowy.
        \item Strony ustalają kaucję w wysokości jednomiesięcznego czynszu.
        \item W związku z § 4 pkt 4 Najemca zobowiązuje się pokryć ewentualne roszczenia z tytułu zniszczenia lokalu ponad normalne zużycie w trakcie używania. Wysokość roszczeń zostanie ustalona na podstawie faktur lub rachunków powykonawczych usuwających zniszczenia.
 	    \item Najemca pokrywa koszty eksploatacji lokalu obejmujące:
            \subitem a) opłaty za gaz - płatne według wskazań liczników,
            \subitem b) opłaty za centralne ogrzewanie - płatne według wskazań liczników,
            \subitem c) opłaty za energię elektryczną - płatne według wskazań liczników,
            \subitem d) opłaty za wodę - płatne według wskazań liczników,
        \item Należności wymienione w § 4 pkt 6 Najemca będzie płacić 	Wynajmującemu na podstawie faktur lub rachunków wystawionych przez dostawców, w terminie do siedmiu dni roboczych od daty doręczenia przez Wynajmującego Najemcy faktur lub rachunków.
        \item W przypadku zwłoki w płatności czynszu i opłat eksploatacyjnych Najemca jest obowiązany do zapłaty odsetek za zwłokę w ustawowej wysokości oraz odsetek stosowanych przez dostawców mediów.
    \end{enumerate}
\section*{§ 5.}
    \begin{enumerate}
        \item Najemcy nie wolno dokonywać zmiany przeznaczenia lokalu.
        \item Najemcy nie wolno oddawać lokalu w podnajem lub do bezpłatnego używania.
        \item Ewentualne planowane przez najemcę adaptacje lub przebudowa pomieszczeń muszą 	być za każdym razem uzgadniane z Wynajmującym i wymagają jego 	pisemnej zgody.
        \item Najemca może wprowadzić w lokalu ulepszenia tylko za pisemną zgodą Wynajmującego.
    \end{enumerate}
\section*{§ 6.}
    \begin{enumerate}
        \item 1. Remonty o charakterze generalnym obciążają Wynajmującego.
        \item Najemca jest obowiązany do dokonywania na własny koszt konserwacji i napraw niezbędnych do zachowania przedmiotu najmu w stanie niepogorszonym.
        \item Najemca zobowiązany jest poinformować Wynajmującego niezwłocznie o konieczności przeprowadzenia napraw i innych czynności, które zgodnie z obowiązującymi przepisami obciążają Wynajmującego.
        \item Najemca zobowiązany jest poinformować Wynajmującego niezwłocznie o konieczności wymiany wkładki do drzwi wejściowych w przypadku 	zgubienia lub zniszczenia któregokolwiek kompletu kluczy, który Najemca otrzymał w dniu podpisania umowy od Wynajmującego.
        \item Najemca zobowiązany jest udostępnić Wynajmującemu lokal do wglądu.
\end{enumerate}
\section*{§ 7.}
    \begin{enumerate}
        \item Po ustaniu stosunku najmu Najemca zobowiązany jest niezwłocznie zwrócić przedmiot najmu Wynajmującemu w stanie niepogorszonym.
        \item Najemca nie ponosi odpowiedzialności za zmiany i zniszczenia w 	lokalu powstałe na skutek naturalnego zużycia.
    \end{enumerate}
\section*{§ 8.}
Wynajmujący nie odpowiada za szkody wyrządzone ruchomościom Najemcy w wyniku zalania, ognia, włamania i innych zdarzeń losowych.
\section*{§ 9.}
W sprawach nieuregulowanych niniejszą umową zastosowanie mają przepisy Kodeksu cywilnego oraz ustawy o ochronie praw lokatorów.
\section*{§ 10.}
Zmiana niniejszej umowy wymaga formy pisemnej pod rygorem nieważności.
\section*{§ 11.}
Umowę sporządzono w pięciu jednakowo brzmiących egzemplarzach, cztery egzemplarzy dla Najemcy i jeden dla Wynajmującego.
\newline
\vspace{40pt}
\Large{\textbf{Najemca}} \\
Jarosław Mirek, zenek@gmail.com, 555777888
\newline
\vspace{60pt}
\Large{\textbf{Wynajmujący}}\\
Jarosław Mirek, zenek@gmail.com, 666555444
\end{document}
